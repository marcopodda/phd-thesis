\chapter{Conclusions and Future Works} % Main chapter title
\label{ch:conclusions}

This thesis has discussed Deep Learning on graphs and two relevant applications of graph-based Deep Learning to the domain of life sciences. The main theme of the thesis has been that Deep Learning methodologies can be an effective tool to adopt when working on life sciences problems, where tasks require powerful learning capabilities coupled with the ability to work with structured data such as graphs. Our focus has been on two relevant sub-domains of life sciences: computational biology and computational chemistry. Each of them presented very different challenges: in computational biology, we focused on supervised (predictive) learning, while in computational chemistry, the focus was on unsupervised (generative) learning.

The first contribution of our thesis has been an extensive and fair evaluation of Deep Graph Network models for graph classification tasks. Our motivation stems from how the comparison among models for graphs has been conducted. We identified several criticalities in the evaluation protocols, such as poor standardization across different works in terms of data splits used, ambiguous or under-specified selection of hyper-parameters, as well as unfair comparison in terms of features used by the different models under evaluation. We thus proceeded to re-evaluate 5 state-of-the-art models in the literature across 9 different datasets, of which 4 are related to the classification of chemical properties of graphs, and 5 concern predictive tasks on social networks. We thus devised a standardized evaluation framework consisting of an internal hold-out based model selection, and an external 10 fold cross validation for model assessment. Importantly, we take all the necessary precautions to design the framework as rigorously and farily as possible: all models are trained on exactly the same stratified data splits, and all models have been assigned a congruent amount of hyper-parameters to optimize during model selection. One important aspect of our work is that the graph models are compared to structure-agnostic baselines, evaluated under the same standardized framework. The purpose of this comparison is to highlight cases where the Deep Graph Networks still struggle to exploit graph structure to the fullest, and to put their results under the correct perspective. Our result differ from those reported in the literature, and are more consistent to what would be expected from a properly carried out comparison. Concerning the comparison with the baselines, we show that in some cases, Deep Graph Networks do not perform better than structure-agnostic baselines. In these cases, we recommend care to not over-emphasize small performance improvements, which are more likely due to be attributed to chance. Finally, we show how a fair comparison allows a reasoned analysis of certain properties of Deep Graph Networks. We do so by demostrating how adding the degree of the graph vertices as a feature translates to a performance improvement across the social datasets. In principle, our analysis can be extended indefinitely by further works, by including more Deep Graph Networks in the comparison, by enlarging the grids of hyperparameters each model can optimize, and by refining the internal model selection phase using cross validation for a less uniased estimation.

Our second contribution has been a novel application of Deep Graph Networks for graph classification to a relevant problem in computational biology. In particular, we studied the problem of predicting the dynamical properties of biological pathways represented as graphs. Our focus has been on the property of robustness, which quantifies the stability of the dynamical system specified by the pathway under perturbations of the initial conditions. Differently from traditional methods, which require to compute the robustness via expensive numerical simulations, we hypothesized that the structure of the graph representing the pathway correlates to its robustness, and that this correlation could be exploited for predictive purposes. The potential implication of this assumption is that, in order to assess robustness or other dynamical properties, one is no more forced to perform the costly numerical simulations. Guided by this motivation, we developed a learning framework in which the structure of a biological pathway is given as input, and an indicator of whether the pathway is robust or not is predicted as output. The framework is composed of a Deep Graph Network that processes the pathway structure, and a downstream Multi-Layer Perceptron classifier which makes the actual robustness prediction. We tested this framework on a dataset of real-world biological pathways taken from the BioModels database. The experimental results validate our initial assumption to a remarkably good extent, paving the way for the use of learning methodologies to assess dynamical properties of biochemical pathways.

In the third part of this thesis, we focused on the problem of graph generation. This is an impacting problem that is being actively researched because of its vaste implications in several domains of applications. Our contribution is a novel model for unlabeled graph generation based on modeling a graph distribution of interest autoregressively. Existing autoregressive approaches model graph generation as in terms of sequences of nodes. We adopt a related but different formulation, which generates the graph according to its sequence of edges ordered lexicographically. The autoregressive distribution is modeled by splitting the ordered edge sequence in two sequences: the first obtained from the starting nodes of the edges, and the second analogously from the ending nodes. The graph is generated by first sampling a sequence of starting nodes with an autoregressive \gls{rnn}, which is ultimately completed into an actual graph by a second \gls{rnn} that complements the starting sequence with the corresponding end sequence. We assess the performances of the proposed models on 5 different datasets, of which 3 are synthetic datasets of graphs with strong node/edge dependencies, and 2 are chemical datasets of molecules. The experimental evaluation has been conducted taking into account quantitative (\ie how many useful graphs the model is able to produce) as well as qualitative (\ie how well the generated graphs resemble the training samples) aspects. The results hows that the model well approximates the desired graph distributions in all cases, performing on par with state-of-the-art approaches. The model also has limitations, in that it is not able (in its current form) to handle labeled generation, which is an obvious topic to investigate in subsequent works. Another interesting direction to take in the next future regards the application of attention mechanisms for better integrating the information of the first \gls{rnn} into the second. More generally, generative models of graphs still need to be researched further and in-depth. Among the many topics, one with potentially positive implications to the advancement of the field is the development of efficient permutation-invariant decoders.

Lastly, we present an application of Deep Generative models of graphs in the context of molecular generation. Generating molecules is an extremely important phase of the drug discovery process, as it allows to cut down the time and costs needed to produce candidate drugs which can advance to further screening phases. Approaches to the generation of molecules are based on two big model families: one linearizes the molecular graphs in SMILES notation, and learn a language model of strings in the SMILES language. The learned language model is finally sampled to produce novel SMILES strings, which correspond to a novel candidate molecule being generated. The other family of models works directly on the molecular graph, generating its structure either at once or sequentially by incrementally adding nodes to an existing graph. This approach is arguably considered more powerful, as it leverages a more expressive representation of the molecule. In comparison, the SMILES-based approach trades off a superior speed in both training and generation phases, with problems caused by the production of chemically invalid and duplicate molecules. In our work, we address these two limitations specifically. To solve the problems related to chemical validity, we adopt a sequential generative approach based on chemical fragments. Chemical fragments are small compounds that can be combined together to form more complex and effective molecules. The proposed model generates molecules as sequences of SMILES fragments, rather than SMILES characters. This has two positive effects. Firstly, it reduces the length of the generating sequence, which implies faster computation and avoidance of long-term dependency problems. Secondly, it lowers the chances for the model to make wrong predictions that can undermine the chemical validity of the generated graph. To solve problems regarding the generation of duplicate (not unique) molecules, we propose a strategy termed Low-Frequency Masking, which is designed to encourage the model to generate molecules with fragments that appear less frequently in a dataset. We test our model on two important chemical dataset benchmarks, evaluating it both quantitatively an qualitatively. The results show that our model dramatically improves upon SMILES-based models in the generation of valid and unique molecules, reaching a level of performances that is typical of the more expressive family of graph-based generative models. However, the model as-is has limited applicability in subsequent generative tasks such as molecular optimization, due to the stochasticity of the Low-Frequency Masking technique. We see at least two promising directions to explore in the near future as regards this work. The first is to improve the model for subsequent tasks like molecular optimization: this requires to devise more effective strategies to avoid the production of duplicate molecules. The second is to combine the fragment-based approach with graph-based methodologies: this has the potential of being a \quotes{best of both worlds} solution exploiting the advantages of fragment-based generation in terms of speed of training/sampling, and the expreviness given by the graph-based representation of the chemical fragments.
\\
In conclusion, we believe the results presented in this thesis demonstrate the flexibility of Deep Learning to \emph{a}) handle complex, variable-sized data structures, and \emph{b}) leverage the expressiveness of these data to give important contributions in the two domains of application.